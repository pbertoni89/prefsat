\section{Livello \emph{short-term memory}}
	Versione base della tecnica \emph{TS}, mira a consentire mosse peggiorative nel tentativo di fuoriuscita dagli ottimi locali. 
	Invoca pertanto una o più esecuzioni del livello \emph{LOS}, secondo determinati parametri \emph{STM} ottenuti da \emph{LTM} e secondo il comportamento
	delle esecuzioni stesse.\\
	Per ogni iterazione di \emph{STM} ogni veicolo ha diritto ad effettuare una e una sola mossa. Una mossa di un veicolo è un compartimento stagno rispetto allo stato
	degli altri veicoli, per la non interazione tra essi discussa alla sezione \ref{subsec:vehicleinter}. \\ 
	Il livello \emph{short-term} ha due funzioni principali, esposte di seguito.
\subsection{Taboo List} %~~~~~~~~~~~~~~~~~~~~~~~~~~~~~~~~~~~~~~~~~~~~~~~~~~~~~~~~~~~~~~~~~~~~~~~~~~~~~~~~~~~~~~~~~~~~
	È una struttura dati di dimensione dinamica contenente le mosse vietate, cioè \emph{forbidden}. Essa viene gestita e aggiornata dal \emph{STM}, ma dev'essere
	nota al \emph{LOS}, il quale non può individuare soluzioni vietate dalla \emph{TL} a meno che non soddisfino i \emph{criteri d'aspirazione}.
	A causa della non interazione tra i veicoli ogni veicolo possiede la \emph{propria} lista e solo ad essa è soggetto.
	Ogni elemento di \emph{TL} è un arco (quindi identificato da due indici $(i,j)$ pari ai suoi nodi) a cui vengono associati due valori:
	\begin{itemize} 
	  \item $\lambda_{ij}^k \in \mathbb{N}\;\;\;$ \emph{taboo tenure}, è il numero di iterazioni per cui $e_{ij}$ non può essere incluso nell'esplorazione d'intorno durante una mossa;
	  \item $\nu_{ij}^k \in \mathbb{N}\;\;\;$ \emph{iterazione corrente} al momento dell'inserzione in \emph{TL} (al limite posta a 0 in inizializzazione)
	\end{itemize}
	e tali per cui $\lambda_{ij}^k+\nu_{ij}^k = \pi_{ij}^k$ \emph{iterazione di expiration} dalla \emph{TL}.
	La semantica è: $e_{ij}$ non può essere considerato nella variazione d'intorno finchè non sarà raggiunta l'iterazione $\pi$; \emph{e.g.}:
	\begin{itemize}
	  \item se durante una mossa l'arco $e_{ij}$ già appartiene a $P_k$, esso non potrà uscire (a prescindere dal tipo di mossa, che si vedrà più avanti)
	  \item  viceversa archi rimossi dalla soluzione non potranno esser considerati come candidati a rientrarvi, sempre in ogni tipo di mossa.
	\end{itemize}
	Pertanto un arco entrante in \emph{TL} alla generica iterazione $\psi$ assume $\pi_{ij}^k\leftarrow\psi+\overline{\lambda}$.
	A seconda del tipo di spostamento (arco aggiunto in $P_k$ oppure rimosso) viene associato un diverso valore $\overline{\lambda}$,
	 parametrizzato dall'utente come descritto in \ref{sec:cfg}.	
	\begin{algorithm}
	 \SetAlgoLined	 
	 \SetKwProg{Fn}{Function}{ is}{end}
	 \Fn{\upshape{isTaboo($\pi_{ij}^k$) :} bool}{ \eIf{$\pi_{ij}^k\leq\psi$}{return false\;}{return true\;} }
	\caption{Procedura per conoscere la condizione di appartenenza di un arco a una Taboo List.}
	\end{algorithm}
	
\subsection{Intensificazione della ricerca} %~~~~~~~~~~~~~~~~~~~~~~~~~~~~~~~~~~~~~~~~~~~~~~~~~~~~~~~~~~~~~~~~~~~~~~~~
\label{subsec:neighbourhoods}
	È mirata a decidere, sulla base di alcuni parametri, la tipologia dell'intorno nel quale la prossima soluzione può trovarsi.
	Classificare gli intorni significa restringere l'intorno originario $\mathcal{N}(x)$ della soluzione $x$, per contenerne le
	dimensioni e soprattutto \emph{intensificare} la ricerca verso una direzione precisa. Si vedrà come tutte le tipologie individuino 
	intorni di dimensione polinomiale, rispettando uno dei vincoli progettuali principali di un'euristica. 
	La logica di intensificazione è il vero focus sulle prestazioni offerte da \emph{STM}.
	Qualora venga deciso di aggiungere un arco a un percorso:
	\begin{itemize} 
		\item che non è stato raccolto da nessuno;
		\item tale per cui la sua raccolta mantiene la soluzione $Q_{OK}$
	\end{itemize}
	esso viene automaticamente attraversato e raccolto, altrimenti semplicemente attraversato.
	
	Sono ora elencate le possibili mosse progettate e implementate. Esse \emph{non} sono computate all'interno di \emph{STM}, ma solo selezionate
	di volta in volta, secondo la \emph{logica di intensificazione}. La vera esecuzione delle mosse, cioè la generazione e l'esplorazione degli intorni, avviene a livello \emph{LOS}.
	
	Si noti che l'\emph{intensificazione} qui trattata non coincide con quella discussa in \emph{LTM}, sebbene il concetto padre sia il medesimo.
\subsubsection{Mossa add} 
	E' definibile come l'intorno $\mathcal{N}_{ADD}(x,g)$ che aggiunga un nuovo nodo al percorso $P_k$, inserendo due nuovi archi da percorrere
	e raccogliere se possibile. I due archi possono essere un \emph{arco doppio}, come in \ref{fig:addScenario1}. 
	\maxfig{add1.png}{addScenario1}{Esempio di mossa \emph{add} orientata alla doppia percorrenza di un arco.}Il dominio di $\mathcal{N}_{ADD}$ contiene tutti e soli gli archi di $P_k$.
	Ognuno di questi individua per costruzione due nodi alle sue estremità, e questi a loro volta individuano $|V\setminus P_k|$ possibili triangolazioni con un nodo
	non appartenente al percorso, oltre a $2|V\setminus P_k|$ archi doppi con i medesimi. La vastità dell'intorno è data quindi dal
	 cartesiano tra gli archi di $P_k$ e
	i nodi di $V \setminus P_k$ candidati all'inserimento, restando quindi simile a una legge quadratica.
	In più, si considerano candidati anche tutti i nodi di $P_k$ eccetto i due nodi estremi che fungon da perno, come in figura \ref{fig:addScenario2}. \\
	\maxfig{add2.png}{addScenario2}{Esempio di mossa \emph{add} orientata verso un nodo già appartenente al percorso}
	Questo porta la complessità finale ad avere un upper bound di $3|E(P_k)-2||V\setminus P_k|$.
	\begin{algorithm}
	 \SetAlgoLined
	 \KwData{$J_k,\; P_k$ feasible}
	 \KwResult{$J_k^\prime,\; P_k^\prime$ feasible}
	 \tcc{insieme degli archi del percorso $k$}
	 \ForEach{$e_{ij}\in P_k$}{
	 	\tcc{insieme degli archi doppi costruiti sul nodo sinistro unito a quelli sul nodo destro}
	 	\ForEach{$e\in \{ e_{il}\notin P_k\}\cup\{ e_{jm}\notin P_k\}$}{
	 		$P_k^\prime \leftarrow P_k \cup \{2\times e\}$\; %\setminus \{e_{ij}\}$\;
	 		\If {\upshape{isTQok($P_k^\prime$)}} {
				\tcc{è tutto fuori da tabu}
				\If {\upshape{\bfnot \,\,isTaboo($e_{ij},\,k$)\,\, \bfand \,\, \bfnot\,\,isTaboo($e,\,k$)}} {
					\tcc{la valutazione è pura o smart, a scelta}
					$J_k\prime \leftarrow$ evaluate($P_k^\prime$)\;

					\If {$J_k\prime > J_k$} {
						$P_k \leftarrow P_k^\prime$\;
						$J_k \leftarrow J_k\prime$\;
					}
				}
	 		}
	 	}
	 	\tcc{insieme degli archi triangolanti sui nodi nuovi e vecchi}
	 	\ForEach{$(e_{il},e_{jl}) \in E$}{
	 		$P_k^\prime \leftarrow P_k \cup \{e_{il}, e_{jl}\} \setminus \{e_{ij}\}$\;
	 		\If {\upshape{isTQok($P_k^\prime$)}} {
				\If {\upshape{\bfnot \,\,isTaboo($e_{ij},\,k$)\,\, \bfand \,\, \bfnot\,\,isTaboo($e_{il},\,k$) \,\, \bfnot\,\,isTaboo($e_{jl},\,k$)}} {
					$J_k\prime \leftarrow$ evaluate($P_k^\prime$)\;

					\If {$J_k\prime > J_k$} {

						$P_k \leftarrow P_k^\prime$\;
						$J_k \leftarrow J_k\prime$\;
					}
				}
	 		}
	 	}
	}
	\caption{Mossa ADD.}
	\end{algorithm}

\subsubsection{Mossa swap} %~~~~~~~~~~~~~~~~~~~~~~~~~~~~~~~~~~~~~~~~~~~~~~~~~~~~~~~~~~~~~~~~~~~~~~~~~~~~~~~~~~~~~~~~~
	È una mossa intesa a scambiare due archi con altri due archi aventi in comune due nodi coi primi. Formalmente quindi il
	dominio di $\mathcal{N}_{SWAP}(x,g)$ è $\{(e_{ij},\,e_{kl}) \in P_k\;\;|\;\; e_{ij} \neq e_{kl}\}$, ovvero tutte le combinazioni senza ripetizione degli
	archi camminati nel percorso. Per ogni coppia di questo insieme si valuta $\tilde P_k$ ottenuto sostituendo ad essa la coppia $(e_{il},\,e_{jk})$ calcolandone la $\mathcal{J}(g\prime)$
	associata. \\
	\maxfig{swap1.png}{swapScenario2}{Esempio di mossa \emph{swap}.}
	La grandezza dell'intorno $\mathcal{N}_{SWAP}$ è quindi quella delle combinazioni di due elementi senza ripetizione dei $|E(P_k)|$ archi del percorso, 
	perciò applicando la binomiale si ottiene $\frac{(E(P_k)-1)\,E(P_k)}{2}$ ovvero una complessità quadratica.
	Nell'elaborazione della mossa si rende necessario un controllo di connessione del percorso per ogni possibile candidato, in quanto uno \emph{swap} 
	tra due archi appartenenti a un ciclo nel $50\%$ dei casi genera due sub-tour. La mossa \emph{swap} diventa perciò un bottleneck teorico nell'esecuzione
	del programma, ma i tempi finali accettabili l'hanno comunque validata come mossa tollerabile.
	\begin{algorithm}
	 \SetAlgoLined
	 \KwData{$J_k,\; P_k$ feasible}
	 \KwResult{$J_k^\prime,\; P_k^\prime$ feasible}
	 \tcc{insieme delle combinazioni senza ripetizione degli archi non contigui del percorso $k$}
	 \ForEach{$(e_{ij},e_{kl})\in P_k \;|\; j\neq l,\,i\neq k,\,e_{ij}\neq e_{kl}\;$}{
 		\tcc{è tutto fuori da tabu, oppure è tutto concesso dai criteri aspirazione}
 		\If {\upshape{\bfnot\,\,isTaboo($e_{ij},\,k$)\,\,\bfand \,\,\bfnot\,\,isTaboo($e_{kl},\,k$)\,\,\bfand\,\,
 					  \bfnot\,\,isTaboo($e_{il},\,k$)\,\,\bfand\,\,\bfnot\,\,isTaboo($e_{jk},\,k$)\,\,\bfand \\
					  \,\,isPathConnected($P_k^\prime$)}} {
 			$P_k^\prime \leftarrow P_k \cup \{e_{il},e_{jk}\} \setminus \{e_{ij},e_{kl}\}$\;
 			\If {\upshape{isTQok($P_k^\prime$)}} {
				\tcc{la valutazione è pura o smart, a scelta}
				$J_k\prime \leftarrow$ evaluate($P_k^\prime$)\;

				\If {$J_k\prime > J_k$} {
					$P_k \leftarrow P_k^\prime$\;
					$J_k \leftarrow J_k\prime$\;
				}
			}
 		}
	}
	\caption{Mossa SWAP.}
	\end{algorithm}

\subsubsection{Mossa remove} %~~~~~~~~~~~~~~~~~~~~~~~~~~~~~~~~~~~~~~~~~~~~~~~~~~~~~~~~~~~~~~~~~~~~~~~~~~~~~~~~~~~~~~~
	Si definisce così una mossa che elimini $2$ archi adiacenti
	$e_{il},\,e_{jl},\,\, i\neq j$ da $P_k\setminus v_1$ e che inserisca in $P_k$ l'arco $e_{ij}$ connettente i due nodi esterni agli archi eliminati, 
	detto \emph{arco triangolante}. In questo senso, la mossa è esattamente speculare all'\emph{add}.
	L'intorno esplorato $\mathcal{N}_{REM}(x,g)$ è perciò l'insieme di tutte le possibili suddette coppie, la rimozione delle quali induce soluzioni 
	$(x\,\prime,g\,\prime)$ che computano $f(g\,\prime)$, essendo la nuova soluzione uguale a $P_k$ privato degli archi semplici ed unito all'arco triangolante.
	
	Come in \emph{swap}, una mossa \emph{remove} può generare la presenza di subtours. Anch'essa dunque effettua un controllo di connessione e valgono le stesse
	considerazioni sull'accettabilità delle performances.
	
	Remove è una mossa progettata per rilassare la quantità di risorse a
	disposizione. Non è sempre possibile, in quanto sebbene per i tempi di percorso vale la disuguaglianza triangolare, per le quantità di servizio 
	il programma non prevede di poter attraversare l'arco triangolante senza raccogliere, qualora la raccolta sia possibile.
	La piena efficienza di remove avviene quando l'elemento selezionato nell'intorno è un'istanza di mossa che riduce più possibile le risorse spese,
	a fronte di un esiguo decremento del profitto come effetto collaterale.
	\begin{algorithm}
	 \SetAlgoLined
	 \KwData{$J_k,\; P_k$ feasible}
	 \KwResult{$J_k^\prime,\; P_k^\prime$ feasible}
	 \ForEach{$e_{ij}\in P_k$}{
	 	\tcc{insieme degli archi triangolanti sui nodi nuovi e vecchi}
	 	\ForEach{$(e_{il},e_{jl}) \in E$}{
	 		$P_k^\prime \leftarrow P_k \cup \{e_{il}, e_{jl}\} \setminus \{e_{ij}\}$\;
	 		\If {\upshape{isTQok($P_k^\prime$)}} {
				\If {\upshape{\bfnot \,\,isTaboo($e_{ij},\,k$)\,\, \bfand \,\, \bfnot\,\,isTaboo($e_{il},\,k$) \,\, \bfnot\,\,isTaboo($e_{jl},\,k$)\,\,\bfand \\
					 \,\,isPathConnected($P_k^\prime$)}} {
					$J_k\prime \leftarrow$ evaluate($P_k^\prime$)\;

					\If {$J_k\prime > J_k$} {

						$P_k \leftarrow P_k^\prime$\;
						$J_k \leftarrow J_k\prime$\;
					}
				}
	 		}
	 	}
	}
	\caption{Mossa REMOVE.}
	\end{algorithm}
	
\subsubsection{Logica di intensificazione} %~~~~~~~~~~~~~~~~~~~~~~~~~~~~~~~~~~~~~~~~~~~~~~~~~~~~~~~~~~~~~~~~~~~~~~~~~
	Presentate quindi le tipologie di mossa occorre definire una logica di livello \emph{STM} che decida come ordinarle e quando richiederle.
	Siccome i percorsi inizializzati dall'euristica costruttiva sono molto brevi, è
	ragionevole tentare di espanderli con \emph{add} in una prima fase.
	Fare più \emph{add} possibili fino a raggiungere i vincoli $TQ_{OK}$ sarebbe però una decisione greedy: si arriverebbe in uno stato dove solo intorni molto piccoli di mosse \emph{add} e \emph{swap} sono possibili, impoverendo notevolmente l'evoluzione.
	Servono dunque delle quantità variabili in uno \emph{spazio di stato delle risorse} $\Psi \triangleq [0,T_{MAX}]\times[0,Q]$, che descrivano 
	la propensione di \emph{STM} a scegliere uno dei tre intorni piuttosto che gli altri, ad ogni iterazione $I_{STM}$. Due ottime candidate sono già state 
	individuate in $\tau^k$ e $\theta^k$.  
	Per sincerarsi da eventuali errori modellistici che potrebbero, forzando deterministicamente lo stesso comportamento, arrestare la convergenza 
	a ottimi locali ben lontani dal globale, è stato ritenuto opportuno rendere probabilistica la scelta del tipo d'intorno, mirando a perturbare 
	possibili decisioni non intelligenti.
	Si cerca quindi una relazione $$ s(\tau,\theta) \, : \,\Psi \subset \mathbb{N}^2 \rightarrow 
	(\alpha_{ADD},\alpha_{SWAP},\alpha_{REM}) = (\alpha_1,\alpha_2,\alpha_3)\in(0,1]^3$$ 
	che leghi le due variabili descritte a tre coefficienti $\alpha$ rappresentanti tre aree probabilistiche dell'universo $\Omega$ delle possibilità; 
	\emph{i.e.} tanto maggiore $\alpha_{ADD}$ rispetto agli altri $\alpha$, tanto più probabilmente verrà in quella iterazione eseguita $\mathcal{N}_{ADD}$.
	Pertanto gli $\alpha$ vanno rapportati in modo percentuale: si creano i valori $\tilde{\alpha_i}=\frac{\alpha_i}{\alpha_1+\alpha_2+\alpha_3}\;\;\; i=1,2,3$.
	Finalmente un numero pseudocasuale distribuito uniformemente tra $0$ e $1$ associa biunivocamente le proprie estrazioni con le tre casistiche.
	La relazione prima accennata è stata modellizzata con un vettore di tre funzioni gaussiane, del tipo
	$$f_X(x) = \frac{1}{{(2\pi)}^3\sqrt{\det C}}a^{-\frac{1}{2}{(x-\mu_X)}^TC^{-1}(x-\mu_X)} $$
	essendo \begin{itemize}
	  \item $x$ il vettore $(\tau,\,\theta)$
	  \item $C$ una matrice di covarianza simmetrica definita positiva
	  \item $\mu_X$ il punto di massimo per la gaussiana.
	  \item $a$ un numero reale di cui si discuterà per ultimo.
	\end{itemize}
	Per prima cosa si è posta la locazione delle tre medie rispettivamente in
	\begin{itemize}
	  \item $\mu_{ADD} = (0,0)\,$, stato di risorse completamente libere
	  \item $\mu_{SWAP} = (\frac{T_{MAX}}2,\frac{Q}2)\,$, stato di risorse mediamente allocate
	  \item $\mu_{REM} =(T_{MAX},Q)\,$, stato di risorse completamente esaurite.
	\end{itemize}
	Non così banale è invece l'assegnazione delle tre matrici di covarianza. Valori bassi potrebbero concentrare le campane in regioni molto piccole
	dello spazio di stato delle risorse, lasciando vaste zone a concentrazione infinitesima e verosimilmente vanificando la complessità di questo approccio.
	Inoltre occorre ponderare la correlazione tra le variabili, per riempire adeguatamente lo spazio di stato $\Psi$ (un reticolo di un parallelepipedo) nel
	caso di problemi con grande disuguaglianza di risorse disponibili.
	Si è seguito un approccio tale per cui, grazie al grado di libertà offerto dalla scelta di $a$, è permesso impostare funzioni gaussiane corrette che
	rispettino la seguente condizione: \\
	sia $b$ il punto di minimo desiderato della curva gaussiana in $\Psi$, allora 
	$$ \log a = \frac{-2\log b}{T^2\,C^{-1}_{11} + TQ\, (C^{-1}_{21} + C^{-1}_{12}) + Q^2\,C^{-1}_{22}} $$
	Ora è permesso settare $C$ a un valore qualsiasi, si è scelto banalmente la matrice identità. Per costruzione pertanto $f_X(x) \geq b \;\;\forall x \in \Psi$.
	
	\maxfig{gaussians.png}{gaussBells}{Esempio di campane \emph{ADD}, \emph{SWAP}, \emph{REMOVE} nello spazio di stato $T, Q$.}
	
	Si osserva infine che, considerato
	\begin{itemize}
	  \item il determinismo \emph{interno} a ciascuna delle tre mosse
	  \item la possibilità che una mossa fallisca, perchè nessuna soluzione tra le sue candidate è $TQ_{OK}$
	\end{itemize}
	invocare la mossa dello stesso tipo di una appena eseguita che abbia restituito un valore di fail è completamente inutile.
	Un piccolo improvement in questi casi è ottenuto con una struttura denominata informalmente \emph{meta taboo list}, che fondamentalmente tiene
	memoria della possibile ultima tipologia di mossa fallita. Il non determinismo dell'estrazione nella logica \emph{STM} permetterà di selezionare
	velocemente un'altra mossa da invocare.