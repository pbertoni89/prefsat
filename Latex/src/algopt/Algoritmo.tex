L'algoritmo proposto si basa fondamentalmente sulla tecnica della Tabu Search, caratterizzata dalla capacità di continuare la ricerca oltre i minimi locali.
Innanzitutto, per poter sfuggire ai suddetti minimi, che arresterebbero un
algoritmo puramente migliorativo, la Tabu Search consente mosse peggiorative.
Per evitare di ricadervi in breve tempo, inoltre, un elenco denominato
\emph{Tabu List} impedisce temporaneamente di effettuare mosse recentemente già eseguite.
Caratteristica basilare della Tabu Search è quindi l'uso sistematico della
memoria: per aumentare l'efficacia del processo di ricerca, viene tenuta traccia non solo delle informazioni locali 
(come il valore corrente della funzione obiettivo), ma anche di alcune informazioni relative all'itinerario percorso. 
Tali informazioni vengono impiegate per guidare la mossa dalla soluzione corrente alla soluzione successiva, 
da scegliersi all'interno del suo \emph{neighbourhood}.

\section{Definizioni e abbreviazioni}
	Si riportano qui termini di uso frequente nella presentazione dell'elaborato. All'inizio di paragrafi e sezioni è possibile trovarli
	in carattere italico per riportare l'attenzione del lettore alla formalità, dopodichè saranno in carattere normale, per non annoiare la lettura.
	\begin{itemize}
	  \item \emph{Grafo}: sovrascrivendo la formale terminologia matematica, con \emph{grafo} ci si riferirà spesso al caso
	  particolare dell'elaborato, quindi a un grafo indiretto di $N$ nodi, completamente magliato, ove un nodo funge da deposito, ove vige
	  la disuguaglianza triangolare sui tempi, ove gli archi oltre a un costo di tempo $t$ possiedono una domanda $d$ e un profitto $p$.
	  \item \emph{Problema}: un oggetto astratto comprendente un \emph{grafo}, un vincolo sul tempo massimo $T$, uno sulla capacità $Q$, e 
	  una quantità fissata di veicoli $K$.
	  \item \emph{Soluzione}: la coppia $(\vec{\vec{x}},\,\vec{\vec{g}})$, rappresentante il numero di attraversamenti e/o raccolti
	  		degli archi. Si noti che il secondo indice delle matrici deriva dall'avere generalmente $K>1$. Per brevità però ci si riferirà 
	  		spesso alla soluzione come semplicemente $(x,\,g)$.
	  \item \emph{Soluzione feasible}: una soluzione che rispetti tutti i vincoli esposti nella modellizzazione.
	  \item \emph{Soluzione} $TQ_{OK}$: una definizione di feasibility sui soli vincoli di tempo e carico. Si applica anche a singole mosse.
	  \item $\tau^k$ indica il tempo attualmente occupato dal $k$-esimo veicolo nel suo tragitto. È sempre pari a $\sum_{(i,j)\in E}x_{ij}^kt_{ij}$
	  \item $\theta^k$ indica la domanda attualmente servita dal $k$-esimo veicolo nel suo raccolto. È sempre pari a $\sum_{(i,j)\in E}g_{ij}^kd_{ij}$
	  \item \emph{Criterio di ottimalità} funzione utilizzata dall'euristica per perseguire la \emph{funzione obiettivo}. A seconda della logica
	  		che verrà adottata durante l'elaborazione, è possibile che il criterio non sempre massimizzi \emph{in senso assoluto} $f(g)$.
	  		Si rimanda alla sottosezione \ref{subsec:eval}.  
	  \item \emph{Intorno/Neighbourhood}: sottoinsieme dello spazio di
	  	soluzioni indotto da
	  		\begin{itemize}
	  		  \item la soluzione attuale del problema
	  		  \item una precisa \emph{forma} dell'intorno tra un insieme finito di forme.
	  		  \item ulteriori costraints come la \emph{Taboo list}.
	  		\end{itemize}
	  		Questa modellizzazione verrà dettagliatamente descritta nelle sue
	  		possibili istanze nella sezione \ref{subsec:neighbourhoods}.
	  \item \emph{Mossa} azione in un preciso intorno volta a massimizzare un \emph{criterio di ottimalità}.
	  \item \emph{Percorso} $P_k$: ciclo sul grafo effettuato dal $k$-esimo
	  veicolo, che inizia e termina nel deposito ed è incluso in una soluzione feasible.
	  \item \emph{Arco doppio}: un arco percorso due volte, avanti e indietro.
	  \item \emph{Taboo Search}: tecnica metaeuristica ben nota in letteratura, per
	  brevità \emph{TS}.
	  \item \emph{Taboo List}: struttura dati chiave della \emph{TS}, per brevità
	  \emph{TL}.
	\end{itemize}
	
%~~~~~~~~~~~~~~~~~~~~~~~~~~~~~~~~~~~~~~~~~~~~~~~~~~~~~~~~~~~~~~~~~~~~~~~~~~~~~~~~~~~~~~~~~~~~~~~~~~~~~~~~~~~~~~~~~~~~~~~~~~~~~~~~~~~~~~~~~~~~~~~~~~
\section{Note preliminari}
\subsection{Magliatura completa del grafo} %~~~~~~~~~~~~~~~~~~~~~~~~~~~~~~~~~~~~~~~~~~~~~~~~~~~~~~~~~~~~~~~~~~~~~~~~~~~~~~~~~~~~~~~~~~~~~~~~~~
	Prevedendo di ricevere in input istanze di grafi non completamente magliati, per conservare le specifiche del problema \emph{UCARPP} li
	si ha completati con i cammini minimi forniti dall'algoritmo di Dijkstra. Quanto detto è valido per i soli costi di tempo del grafo;
	gli archi non forniti in input, ovvero tutti e soli quelli a cui viene assegnato il cammino minimo, vengono creati con profitto e domanda nulli.
\subsection{Criteri di aspirazione} % ~~~~~~~~~~~~~~~~~~~~~~~~~~~~~~~~~~~~~~~~~~~~~~~~~~~~~~~~~~~~~~~~~~~~~~~~~~~~~~~~~~~~~~~~~~~~~~~~~~~~~~~~
Non sono stati previsti al momento criteri di aspirazione per la Taboo List.	
	
\subsection{Il \emph{valutatore}, un criterio per la pesatura degli archi} %~~~~~~~~~~~~~~~~~~~~~~~~~~~~~~~~~~~~~~~~~~~~~~~~~~~~~~~~~~~~~~~~~~
\label{subsec:eval}
	È un modulo funzionale che, noti $(t_{ij},\,p_{ij},\,d_{ij},\,\tau^k,\,\theta^k,\,T,\,Q)$ restituisce uno scalare $w_{ij}^k$ tanto più vicino a $0^+$ 
	quanto l'inclusione di $e_{ij}$ nel $k$ cammino iniziale è sconveniente e tanto tendente a $\infty^+$ quanto invece è interessante. 
	Il metodo lavora in differenti modalità, gestite da un \emph{flag} settato a livello superiore.
	\begin{itemize}
	  \item se la modalità è \emph{pure}, viene meramente restituito il profitto $p_{ij}$ associato all'arco. Questo serve a ricercare archi che migliorino
	  	    la funzione obiettivo in senso stretto, non preoccupandosi del carico di questi sui vincoli di tempo e di capacità.
	  \item se la modalità è \emph{smart}, il profitto viene rapportato ai costi di attraversamento e servizio ad esso connessi, 
		  	sperando di effettuare delle scelte più lungimiranti rispetto ai vincoli di tempo e capacità. Definendo:
			$$w_{ij}^t \triangleq \frac{p_{ij}}{t_{ij}},\;\;\;\; w_{ij}^d \triangleq \frac{p_{ij}}{d_{ij}},\;\;\;\; 
			\bar{\tau}^k \triangleq T-\tau^k,\;\;\;\; \bar{\theta}^k \triangleq Q-\theta^k$$
			si avrà che, se $\bar{\tau}^k > 0\; \wedge\; \bar{\theta}^k > 0$, ovvero l'arco è $TQ_{OK}$, il peso è dato da
			$$w_{ij}^k \triangleq w_{ij}^t + w_{ij}^d$$
	  \item se la modalità è \emph{wise}, la modalità \emph{smart} viene estesa, aggiungendo un peso alle \emph{risorse} libere, cioè effettivamente possedute:
	  		le variabili $\bar{\tau}^k$ e $\bar{\theta}^k$. Sempre in caso $TQ_{OK}$, il peso è dato da
			$$w_{ij}^k \triangleq w_{ij}^t \bar{\tau}^k + w_{ij}^d \bar{\theta}^k$$  		
	\end{itemize}
	\subsubsection{Archi non profittevoli}
	Per costruzione, sono tutti quegli archi non inclusi nel problema iniziale (e quindi magliati con Dijkstra), oltre ad eventuali archi del problema che
	abbiano $p$ e $d$ nulli. Per ipotesi semplificativa non sono ammessi (nè d'altronde presenti nelle istanze di test) dei grafi con archi aventi \emph{un solo}
	valore nullo di questi due. Il criterio di pesatura degli archi non profittevoli ricalca quello generico per i profittevoli, ma è naturalmente limitato al solo $t$
	dell'arco e al $\tau^k$ del veicolo.
	
\subsection{Euristica costruttiva} %~~~~~~~~~~~~~~~~~~~~~~~~~~~~~~~~~~~~~~~~~~~~~~~~~~~~~~~~~~~~~~~~~~~~~~~~~~~~~~~~~~~~~~~~~~~~~~~~~~~~~~~~~~~
\label{subsec:eurconstr}
	È un algoritmo inesatto che, dato il \emph{problema} in ingresso e una \emph{soluzione} vuota, garantisce come postcondizione la
	generazione di una soluzione $TQ_{OK}$ che serva di partenza al resto dell'algoritmo.
	Le versioni ora presentate non sono state progettate come \emph{state-of-art} e potranno essere affiancate o sostituite da
	costruttive migliori e più complesse.
\begin{itemize}
 \item Simple: I $K$ cammini iniziali selezionati sono tutti dei diversi \emph{archi doppi}, tutti sicuramente serviti, colleganti il nodo deposito a altrettanti $K$ vertici. 
  	La selezione tra i $N-1$ candidati avviene cercando di limitare l'intrinseca caratteristica \emph{greedy} della fase costruttiva: il criterio valutatore è quindi in questa
  	fase richiamato in modalità \emph{wise}, descritta precedentemente in \ref{subsec:eval}. Questa euristica è estremamente povera, in quanto si preoccupa principalmente di rispettare il vincolo $TQ_{OK}$ del problema. È d'altronde
	   veramente improbabile che tale vincolo sia sfiorato in cammini composti da archi doppi.
  	\begin{figure}[H] 
	 	\begin{center}\includegraphics[width=4cm]{./images/costruttiva.png} \end{center}
	 \caption{Un'esecuzione tipica dell'euristica costruttiva.} \label{fig:costruttiva}
 	\end{figure}
 \item Simple Random: anche in questo caso si identificano archi doppi a partire dal deposito, ma non viene preso in considerazione il valutatore, con selezione puramente casuale degli archi doppi.
\end{itemize}

\subsection{Note sull'interazione tra i veicoli} %~~~~~~~~~~~~~~~~~~~~~~~~~~~~~~~~~~~~~~~~~~~~~~~~~~~~~~~~~~~~~~~~~~~~~~
\label{subsec:vehicleinter}
	I veicoli sono \emph{concorrenti} e non \emph{competitori}: si assume che appartengano ad un unico ente che abbia il potere decisionale totale su essi
	e la volontà di ottimizzare il profitto totale, senza focalizzarsi su quello del singolo veicolo. Il problema dunque \emph{non} è di \emph{teoria dei giochi}
	e i veicoli non hanno bisogno di influenzarsi a vicenda. La logica seguita è pertanto ``fai il meglio con ciò che hai a disposizione'', ovvero con la disponibilità
	di ogni arco ad essere ancora servito o meno. In questo senso, l'unica struttura dati condivisa tra veicoli è $g$ della soluzione.
	Ulteriori considerazioni saranno descritte nel livello \emph{STM}.
	
	Una tecnica estremamente \emph{brute} e \emph{greedy} vorrebbe una sequenzialità totale tra le $k$ euristiche relative ai veicoli, portando
	di fatto a un primo percorso molto buono e agli ultimi veramente scarsi. Un'altra molto semplice deciderebbe \emph{prima} tutti gli archi
	attraversati nel problema e successivamente il sottoinsieme degli archi raccolti, imponendo enormi costraints peggiorativi sull'ottimo euristico.
	L'approccio seguito è quindi una sequenzialità ristretta alla singola iterazione \emph{i}-esima dell'algoritmo, un \emph{round robin} su quale
	veicolo debba effettuare l'\emph{i}-esima mossa.
%~~~~~~~~~~~~~~~~~~~~~~~~~~~~~~~~~~~~~~~~~~~~~~~~~~~~~~~~~~~~~~~~~~~~~~~~~~~~~~~~~~~~~~~~~~~~~~~~~~~~~~~~~~~~~~~~~~~~~~~~~~~~~~~~~~~~~~~~~~~~~~
\section{Vista generale sull'algoritmo: un approccio bottom-up}
	La progettazione è incentrata sull'incapsulamento di più layer di logica, da quello più essenziale (ricerca di un ottimo locale)
	al livello metaeuristico più astratto. Sono stati individuati tre macrolivelli dell'algoritmo:
	\begin{itemize}
	  \item \textbf{Local optimum search}(\emph{LOS}) euristica semplice per selezionare il \emph{best improvement}
	  di un dato intorno, date le restrizioni imposte da \emph{STM}. Più avanti si vedrà che il \emph{best improvement}
	  può benissimo essere peggiorativo.
	  \item \textbf{Short-term memory}(\emph{STM}) metaeuristica della \emph{TS} che si incarica di intensificare la ricerca di soluzioni 
	  in una direzione, pilotando opportunamente il livello \emph{LOS}. 
	  Custodisce inoltre la Taboo List, memoria a breve termine delle mosse necessaria per permettere alla funzione obiettivo di peggiorare. 
	  \item \textbf{Long-term memory}(\emph{LTM}) metaeuristica della \emph{TS} che si incarica di pilotare il livello \emph{STM}, 
	  per diversificare la ricerca dell'ottimo in altre direzioni nello spazio delle soluzioni.
	\end{itemize}
	\begin{figure}[H] 
	 	\begin{center}
	 	\includegraphics[width=16.7cm]{./images/alg-flow.pdf}
	 	\end{center}
	 	\caption{Diagramma di attività del programma implementato.}
	 	\label{fig:algFlow}
 	\end{figure}
%~~~~~~~~~~~~~~~~~~~~~~~~~~~~~~~~~~~~~~~~~~~~~~~~~~~~~~~~~~~~~~~~~~~~~~~~~~~~~~~~~~~~~~~~~~~~~~~~~~~~~~~~~~~~~~~~~~~~
\section{Livello \emph{long-term memory}}

È il livello di logica più alta del programma. Lo scopo principale del livello è quello di generare soluzioni costruttive
per avviare l'algoritmo specificato in \emph{STM}: infatti l'algoritmo ottiene risultati diversi a seconda della costruttiva.
Due sono i problemi fondamentali che devono essere risolti al livello \emph{LTM}:
\begin{itemize}
	\item come generare la costruttiva da fornire al livello \emph{STM};
	\item determinare il numero di volte di esecuzioni di \emph{STM};
\end{itemize}

\maxfig{ltm.png}{fig:ltm}{Schema generale del livello \emph{LTM}.}

\subsection{Generazione delle soluzioni costruttive}
Il livello \emph{long term memory} genera diverse soluzioni costruttive a seconda del numero d'iterazione $iterSTM$ attuale.
\begin{itemize}
	\item in caso $iterSTM = 1$, la costruttiva viene generata secondo l'algoritmo \emph{Simple} specificato in \ref{subsec:eurconstr};
	\item in caso $iterSTM = 2$, la costruttiva viene generata secondo l'algoritmo \emph{SimpleRandom} specificato in \ref{subsec:eurconstr};
	\item in caso $iterSTM > 2$, si ha una generazione dinamica che tiene conto delle soluzioni ottenute alle precedenti iterazioni di STM.
\end{itemize}
L'ultimo caso si divide ulteriormente a seconda della necessità di \emph{intensificare} o di \emph{diversificare} la soluzione. 

\subsubsection{Intensificazione}
\emph{LTM} inserisce come soluzione costruttiva la stessa soluzione ottima ritornata
da \emph{STM}, senza aggiungere alcuna tenure.

\subsubsection{Diversificazione}

Per generare la soluzione costruttiva in caso di diversificazione ci si appoggia a dei dati raccolti in una tabella delle frequenze. La tabelle delle
frequenze esprime, per ogni arco possibile, il numero di volte che esso e' stato in soluzione: ogni soluzione ritornata
da \emph{STM} viene analizzata in modo da sommare ad ogni valore nella tabella 1 se l'arco è incluso nella soluzione rilasciata, 0 se non lo è.\\
In questo modo è possibile stabilire quali archi siano i più coinvolti nello storico delle soluzioni. Tuttavia la soluzione costruttiva
non ingloba questi archi, ossia quelli più favoriti, bensì gli archi che meno sono apparsi nell'intero storico delle soluzioni.\\
La soluzione generata dal livello è una soluzione in cui:
\begin{itemize}
	\item ogni arco è doppio;
	\item vengono scelti gli archi più rari nell'intero storico delle soluzioni ritornate da \emph{STM};
	\item da un nodo escono zero o più doppi archi
	\item TQOK;
\end{itemize}

Le soluzioni costruttive dunque hanno una topologia ad albero in cui, tuttavia, ogni ramo dell'albero è in realtà un doppio arco.

\subsubsection{Scelta tra intensificazione e diversificazione}
Una problema da risolvere si presenta nel determinare, data una soluzione rilasciata da \emph{STM}, se è necessario intensificare o diversificare. 
Questa decisione viene presa a seconda di quanto la soluzione ritornata da \emph{STM} sia stata già intensificata.\\
Immaginiamo infatti che \emph{STM} esegua al massimo 100 iterazioni e che alla fine ritorni la soluzione $\alpha$. Decidere se intensificare o diversificare dipende praticamente da quanto tempo $\alpha$ è stata soluzione ottima all'interno di \emph{STM}; se infatti $\alpha$ è diventata soluzione ottima, per esempio, alla quinta iterazione è evidente che essa è già stata stressata abbastanza e che quindi è meglio diversificare. Se invece $\alpha$ è diventata soluzione ottima solo alla 99-esima iterazione, \emph{STM} è stata costretta dal vincolo delle
100 iterazioni ad uscire. Invece, magari, con qualche iterazione in più, era possibile migliorare ulteriormente la soluzione: in questo caso
la soluzione va intensificata.\\
Bisogna quindi semplicemente stabilire una soglia di iterazioni \emph{LOS} \emph{a} entro cui:
\begin{itemize}
	\item se la soluzione ottima è tale da meno di \emph{a} allora bisogna intensificare;
	\item altrimenti bisogna diversificare
\end{itemize}
Tale soglia è nell'algoritmo definita come una percentuale del massimo numero di iterazioni \emph{LOS} ammissibili in una esecuzione di \emph{STM}.
Il parametro \textbf{LTM INTENSIFICATION STM THRESHOLD} esprime tale percentuale ed è stato fissato in modo hardcoded a 5\%.

\subsection{\emph{Stopping rule} del livello \emph{LTM}}
\emph{Long term memory} termina quando vengono eseguiti il numero di iterazioni \emph{STM} indicato tra i parametri del file di configurazione. Si rimanda alla sezione \ref{sec:cfg}.

%~~~~~~~~~~~~~~~~~~~~~~~~~~~~~~~~~~~~~~~~~~~~~~~~~~~~~~~~~~~~~~~~~~~~~~~~~~~~~~~~~~~~~~~~~~~~~~~~~~~~~~~~~~~~~~~~~~~~
\section{Livello \emph{short-term memory}}
	Versione base della tecnica \emph{TS}, mira a consentire mosse peggiorative nel tentativo di fuoriuscita dagli ottimi locali. 
	Invoca pertanto una o più esecuzioni del livello \emph{LOS}, secondo determinati parametri \emph{STM} ottenuti da \emph{LTM} e secondo il comportamento
	delle esecuzioni stesse.\\
	Per ogni iterazione di \emph{STM} ogni veicolo ha diritto ad effettuare una e una sola mossa. Una mossa di un veicolo è un compartimento stagno rispetto allo stato
	degli altri veicoli, per la non interazione tra essi discussa alla sezione \ref{subsec:vehicleinter}. \\ 
	Il livello \emph{short-term} ha due funzioni principali, esposte di seguito.
\subsection{Taboo List} %~~~~~~~~~~~~~~~~~~~~~~~~~~~~~~~~~~~~~~~~~~~~~~~~~~~~~~~~~~~~~~~~~~~~~~~~~~~~~~~~~~~~~~~~~~~~
	È una struttura dati di dimensione dinamica contenente le mosse vietate, cioè \emph{forbidden}. Essa viene gestita e aggiornata dal \emph{STM}, ma dev'essere
	nota al \emph{LOS}, il quale non può individuare soluzioni vietate dalla \emph{TL} a meno che non soddisfino i \emph{criteri d'aspirazione}.
	A causa della non interazione tra i veicoli ogni veicolo possiede la \emph{propria} lista e solo ad essa è soggetto.
	Ogni elemento di \emph{TL} è un arco (quindi identificato da due indici $(i,j)$ pari ai suoi nodi) a cui vengono associati due valori:
	\begin{itemize} 
	  \item $\lambda_{ij}^k \in \mathbb{N}\;\;\;$ \emph{taboo tenure}, è il numero di iterazioni per cui $e_{ij}$ non può essere incluso nell'esplorazione d'intorno durante una mossa;
	  \item $\nu_{ij}^k \in \mathbb{N}\;\;\;$ \emph{iterazione corrente} al momento dell'inserzione in \emph{TL} (al limite posta a 0 in inizializzazione)
	\end{itemize}
	e tali per cui $\lambda_{ij}^k+\nu_{ij}^k = \pi_{ij}^k$ \emph{iterazione di expiration} dalla \emph{TL}.
	La semantica è: $e_{ij}$ non può essere considerato nella variazione d'intorno finchè non sarà raggiunta l'iterazione $\pi$; \emph{e.g.}:
	\begin{itemize}
	  \item se durante una mossa l'arco $e_{ij}$ già appartiene a $P_k$, esso non potrà uscire (a prescindere dal tipo di mossa, che si vedrà più avanti)
	  \item  viceversa archi rimossi dalla soluzione non potranno esser considerati come candidati a rientrarvi, sempre in ogni tipo di mossa.
	\end{itemize}
	Pertanto un arco entrante in \emph{TL} alla generica iterazione $\psi$ assume $\pi_{ij}^k\leftarrow\psi+\overline{\lambda}$.
	A seconda del tipo di spostamento (arco aggiunto in $P_k$ oppure rimosso) viene associato un diverso valore $\overline{\lambda}$,
	 parametrizzato dall'utente come descritto in \ref{sec:cfg}.	
	\begin{algorithm}
	 \SetAlgoLined	 
	 \SetKwProg{Fn}{Function}{ is}{end}
	 \Fn{\upshape{isTaboo($\pi_{ij}^k$) :} bool}{ \eIf{$\pi_{ij}^k\leq\psi$}{return false\;}{return true\;} }
	\caption{Procedura per conoscere la condizione di appartenenza di un arco a una Taboo List.}
	\end{algorithm}
	
\subsection{Intensificazione della ricerca} %~~~~~~~~~~~~~~~~~~~~~~~~~~~~~~~~~~~~~~~~~~~~~~~~~~~~~~~~~~~~~~~~~~~~~~~~
\label{subsec:neighbourhoods}
	È mirata a decidere, sulla base di alcuni parametri, la tipologia dell'intorno nel quale la prossima soluzione può trovarsi.
	Classificare gli intorni significa restringere l'intorno originario $\mathcal{N}(x)$ della soluzione $x$, per contenerne le
	dimensioni e soprattutto \emph{intensificare} la ricerca verso una direzione precisa. Si vedrà come tutte le tipologie individuino 
	intorni di dimensione polinomiale, rispettando uno dei vincoli progettuali principali di un'euristica. 
	La logica di intensificazione è il vero focus sulle prestazioni offerte da \emph{STM}.
	Qualora venga deciso di aggiungere un arco a un percorso:
	\begin{itemize} 
		\item che non è stato raccolto da nessuno;
		\item tale per cui la sua raccolta mantiene la soluzione $Q_{OK}$
	\end{itemize}
	esso viene automaticamente attraversato e raccolto, altrimenti semplicemente attraversato.
	
	Sono ora elencate le possibili mosse progettate e implementate. Esse \emph{non} sono computate all'interno di \emph{STM}, ma solo selezionate
	di volta in volta, secondo la \emph{logica di intensificazione}. La vera esecuzione delle mosse, cioè la generazione e l'esplorazione degli intorni, avviene a livello \emph{LOS}.
	
	Si noti che l'\emph{intensificazione} qui trattata non coincide con quella discussa in \emph{LTM}, sebbene il concetto padre sia il medesimo.
\subsubsection{Mossa add} 
	E' definibile come l'intorno $\mathcal{N}_{ADD}(x,g)$ che aggiunga un nuovo nodo al percorso $P_k$, inserendo due nuovi archi da percorrere
	e raccogliere se possibile. I due archi possono essere un \emph{arco doppio}, come in \ref{fig:addScenario1}. 
	\maxfig{add1.png}{addScenario1}{Esempio di mossa \emph{add} orientata alla doppia percorrenza di un arco.}Il dominio di $\mathcal{N}_{ADD}$ contiene tutti e soli gli archi di $P_k$.
	Ognuno di questi individua per costruzione due nodi alle sue estremità, e questi a loro volta individuano $|V\setminus P_k|$ possibili triangolazioni con un nodo
	non appartenente al percorso, oltre a $2|V\setminus P_k|$ archi doppi con i medesimi. La vastità dell'intorno è data quindi dal
	 cartesiano tra gli archi di $P_k$ e
	i nodi di $V \setminus P_k$ candidati all'inserimento, restando quindi simile a una legge quadratica.
	In più, si considerano candidati anche tutti i nodi di $P_k$ eccetto i due nodi estremi che fungon da perno, come in figura \ref{fig:addScenario2}. \\
	\maxfig{add2.png}{addScenario2}{Esempio di mossa \emph{add} orientata verso un nodo già appartenente al percorso}
	Questo porta la complessità finale ad avere un upper bound di $3|E(P_k)-2||V\setminus P_k|$.
	\begin{algorithm}
	 \SetAlgoLined
	 \KwData{$J_k,\; P_k$ feasible}
	 \KwResult{$J_k^\prime,\; P_k^\prime$ feasible}
	 \tcc{insieme degli archi del percorso $k$}
	 \ForEach{$e_{ij}\in P_k$}{
	 	\tcc{insieme degli archi doppi costruiti sul nodo sinistro unito a quelli sul nodo destro}
	 	\ForEach{$e\in \{ e_{il}\notin P_k\}\cup\{ e_{jm}\notin P_k\}$}{
	 		$P_k^\prime \leftarrow P_k \cup \{2\times e\}$\; %\setminus \{e_{ij}\}$\;
	 		\If {\upshape{isTQok($P_k^\prime$)}} {
				\tcc{è tutto fuori da tabu}
				\If {\upshape{\bfnot \,\,isTaboo($e_{ij},\,k$)\,\, \bfand \,\, \bfnot\,\,isTaboo($e,\,k$)}} {
					\tcc{la valutazione è pura o smart, a scelta}
					$J_k\prime \leftarrow$ evaluate($P_k^\prime$)\;

					\If {$J_k\prime > J_k$} {
						$P_k \leftarrow P_k^\prime$\;
						$J_k \leftarrow J_k\prime$\;
					}
				}
	 		}
	 	}
	 	\tcc{insieme degli archi triangolanti sui nodi nuovi e vecchi}
	 	\ForEach{$(e_{il},e_{jl}) \in E$}{
	 		$P_k^\prime \leftarrow P_k \cup \{e_{il}, e_{jl}\} \setminus \{e_{ij}\}$\;
	 		\If {\upshape{isTQok($P_k^\prime$)}} {
				\If {\upshape{\bfnot \,\,isTaboo($e_{ij},\,k$)\,\, \bfand \,\, \bfnot\,\,isTaboo($e_{il},\,k$) \,\, \bfnot\,\,isTaboo($e_{jl},\,k$)}} {
					$J_k\prime \leftarrow$ evaluate($P_k^\prime$)\;

					\If {$J_k\prime > J_k$} {

						$P_k \leftarrow P_k^\prime$\;
						$J_k \leftarrow J_k\prime$\;
					}
				}
	 		}
	 	}
	}
	\caption{Mossa ADD.}
	\end{algorithm}

\subsubsection{Mossa swap} %~~~~~~~~~~~~~~~~~~~~~~~~~~~~~~~~~~~~~~~~~~~~~~~~~~~~~~~~~~~~~~~~~~~~~~~~~~~~~~~~~~~~~~~~~
	È una mossa intesa a scambiare due archi con altri due archi aventi in comune due nodi coi primi. Formalmente quindi il
	dominio di $\mathcal{N}_{SWAP}(x,g)$ è $\{(e_{ij},\,e_{kl}) \in P_k\;\;|\;\; e_{ij} \neq e_{kl}\}$, ovvero tutte le combinazioni senza ripetizione degli
	archi camminati nel percorso. Per ogni coppia di questo insieme si valuta $\tilde P_k$ ottenuto sostituendo ad essa la coppia $(e_{il},\,e_{jk})$ calcolandone la $\mathcal{J}(g\prime)$
	associata. \\
	\maxfig{swap1.png}{swapScenario2}{Esempio di mossa \emph{swap}.}
	La grandezza dell'intorno $\mathcal{N}_{SWAP}$ è quindi quella delle combinazioni di due elementi senza ripetizione dei $|E(P_k)|$ archi del percorso, 
	perciò applicando la binomiale si ottiene $\frac{(E(P_k)-1)\,E(P_k)}{2}$ ovvero una complessità quadratica.
	Nell'elaborazione della mossa si rende necessario un controllo di connessione del percorso per ogni possibile candidato, in quanto uno \emph{swap} 
	tra due archi appartenenti a un ciclo nel $50\%$ dei casi genera due sub-tour. La mossa \emph{swap} diventa perciò un bottleneck teorico nell'esecuzione
	del programma, ma i tempi finali accettabili l'hanno comunque validata come mossa tollerabile.
	\begin{algorithm}
	 \SetAlgoLined
	 \KwData{$J_k,\; P_k$ feasible}
	 \KwResult{$J_k^\prime,\; P_k^\prime$ feasible}
	 \tcc{insieme delle combinazioni senza ripetizione degli archi non contigui del percorso $k$}
	 \ForEach{$(e_{ij},e_{kl})\in P_k \;|\; j\neq l,\,i\neq k,\,e_{ij}\neq e_{kl}\;$}{
 		\tcc{è tutto fuori da tabu, oppure è tutto concesso dai criteri aspirazione}
 		\If {\upshape{\bfnot\,\,isTaboo($e_{ij},\,k$)\,\,\bfand \,\,\bfnot\,\,isTaboo($e_{kl},\,k$)\,\,\bfand\,\,
 					  \bfnot\,\,isTaboo($e_{il},\,k$)\,\,\bfand\,\,\bfnot\,\,isTaboo($e_{jk},\,k$)\,\,\bfand \\
					  \,\,isPathConnected($P_k^\prime$)}} {
 			$P_k^\prime \leftarrow P_k \cup \{e_{il},e_{jk}\} \setminus \{e_{ij},e_{kl}\}$\;
 			\If {\upshape{isTQok($P_k^\prime$)}} {
				\tcc{la valutazione è pura o smart, a scelta}
				$J_k\prime \leftarrow$ evaluate($P_k^\prime$)\;

				\If {$J_k\prime > J_k$} {
					$P_k \leftarrow P_k^\prime$\;
					$J_k \leftarrow J_k\prime$\;
				}
			}
 		}
	}
	\caption{Mossa SWAP.}
	\end{algorithm}

\subsubsection{Mossa remove} %~~~~~~~~~~~~~~~~~~~~~~~~~~~~~~~~~~~~~~~~~~~~~~~~~~~~~~~~~~~~~~~~~~~~~~~~~~~~~~~~~~~~~~~
	Si definisce così una mossa che elimini $2$ archi adiacenti
	$e_{il},\,e_{jl},\,\, i\neq j$ da $P_k\setminus v_1$ e che inserisca in $P_k$ l'arco $e_{ij}$ connettente i due nodi esterni agli archi eliminati, 
	detto \emph{arco triangolante}. In questo senso, la mossa è esattamente speculare all'\emph{add}.
	L'intorno esplorato $\mathcal{N}_{REM}(x,g)$ è perciò l'insieme di tutte le possibili suddette coppie, la rimozione delle quali induce soluzioni 
	$(x\,\prime,g\,\prime)$ che computano $f(g\,\prime)$, essendo la nuova soluzione uguale a $P_k$ privato degli archi semplici ed unito all'arco triangolante.
	
	Come in \emph{swap}, una mossa \emph{remove} può generare la presenza di subtours. Anch'essa dunque effettua un controllo di connessione e valgono le stesse
	considerazioni sull'accettabilità delle performances.
	
	Remove è una mossa progettata per rilassare la quantità di risorse a
	disposizione. Non è sempre possibile, in quanto sebbene per i tempi di percorso vale la disuguaglianza triangolare, per le quantità di servizio 
	il programma non prevede di poter attraversare l'arco triangolante senza raccogliere, qualora la raccolta sia possibile.
	La piena efficienza di remove avviene quando l'elemento selezionato nell'intorno è un'istanza di mossa che riduce più possibile le risorse spese,
	a fronte di un esiguo decremento del profitto come effetto collaterale.
	\begin{algorithm}
	 \SetAlgoLined
	 \KwData{$J_k,\; P_k$ feasible}
	 \KwResult{$J_k^\prime,\; P_k^\prime$ feasible}
	 \ForEach{$e_{ij}\in P_k$}{
	 	\tcc{insieme degli archi triangolanti sui nodi nuovi e vecchi}
	 	\ForEach{$(e_{il},e_{jl}) \in E$}{
	 		$P_k^\prime \leftarrow P_k \cup \{e_{il}, e_{jl}\} \setminus \{e_{ij}\}$\;
	 		\If {\upshape{isTQok($P_k^\prime$)}} {
				\If {\upshape{\bfnot \,\,isTaboo($e_{ij},\,k$)\,\, \bfand \,\, \bfnot\,\,isTaboo($e_{il},\,k$) \,\, \bfnot\,\,isTaboo($e_{jl},\,k$)\,\,\bfand \\
					 \,\,isPathConnected($P_k^\prime$)}} {
					$J_k\prime \leftarrow$ evaluate($P_k^\prime$)\;

					\If {$J_k\prime > J_k$} {

						$P_k \leftarrow P_k^\prime$\;
						$J_k \leftarrow J_k\prime$\;
					}
				}
	 		}
	 	}
	}
	\caption{Mossa REMOVE.}
	\end{algorithm}
	
\subsubsection{Logica di intensificazione} %~~~~~~~~~~~~~~~~~~~~~~~~~~~~~~~~~~~~~~~~~~~~~~~~~~~~~~~~~~~~~~~~~~~~~~~~~
	Presentate quindi le tipologie di mossa occorre definire una logica di livello \emph{STM} che decida come ordinarle e quando richiederle.
	Siccome i percorsi inizializzati dall'euristica costruttiva sono molto brevi, è
	ragionevole tentare di espanderli con \emph{add} in una prima fase.
	Fare più \emph{add} possibili fino a raggiungere i vincoli $TQ_{OK}$ sarebbe però una decisione greedy: si arriverebbe in uno stato dove solo intorni molto piccoli di mosse \emph{add} e \emph{swap} sono possibili, impoverendo notevolmente l'evoluzione.
	Servono dunque delle quantità variabili in uno \emph{spazio di stato delle risorse} $\Psi \triangleq [0,T_{MAX}]\times[0,Q]$, che descrivano 
	la propensione di \emph{STM} a scegliere uno dei tre intorni piuttosto che gli altri, ad ogni iterazione $I_{STM}$. Due ottime candidate sono già state 
	individuate in $\tau^k$ e $\theta^k$.  
	Per sincerarsi da eventuali errori modellistici che potrebbero, forzando deterministicamente lo stesso comportamento, arrestare la convergenza 
	a ottimi locali ben lontani dal globale, è stato ritenuto opportuno rendere probabilistica la scelta del tipo d'intorno, mirando a perturbare 
	possibili decisioni non intelligenti.
	Si cerca quindi una relazione $$ s(\tau,\theta) \, : \,\Psi \subset \mathbb{N}^2 \rightarrow 
	(\alpha_{ADD},\alpha_{SWAP},\alpha_{REM}) = (\alpha_1,\alpha_2,\alpha_3)\in(0,1]^3$$ 
	che leghi le due variabili descritte a tre coefficienti $\alpha$ rappresentanti tre aree probabilistiche dell'universo $\Omega$ delle possibilità; 
	\emph{i.e.} tanto maggiore $\alpha_{ADD}$ rispetto agli altri $\alpha$, tanto più probabilmente verrà in quella iterazione eseguita $\mathcal{N}_{ADD}$.
	Pertanto gli $\alpha$ vanno rapportati in modo percentuale: si creano i valori $\tilde{\alpha_i}=\frac{\alpha_i}{\alpha_1+\alpha_2+\alpha_3}\;\;\; i=1,2,3$.
	Finalmente un numero pseudocasuale distribuito uniformemente tra $0$ e $1$ associa biunivocamente le proprie estrazioni con le tre casistiche.
	La relazione prima accennata è stata modellizzata con un vettore di tre funzioni gaussiane, del tipo
	$$f_X(x) = \frac{1}{{(2\pi)}^3\sqrt{\det C}}a^{-\frac{1}{2}{(x-\mu_X)}^TC^{-1}(x-\mu_X)} $$
	essendo \begin{itemize}
	  \item $x$ il vettore $(\tau,\,\theta)$
	  \item $C$ una matrice di covarianza simmetrica definita positiva
	  \item $\mu_X$ il punto di massimo per la gaussiana.
	  \item $a$ un numero reale di cui si discuterà per ultimo.
	\end{itemize}
	Per prima cosa si è posta la locazione delle tre medie rispettivamente in
	\begin{itemize}
	  \item $\mu_{ADD} = (0,0)\,$, stato di risorse completamente libere
	  \item $\mu_{SWAP} = (\frac{T_{MAX}}2,\frac{Q}2)\,$, stato di risorse mediamente allocate
	  \item $\mu_{REM} =(T_{MAX},Q)\,$, stato di risorse completamente esaurite.
	\end{itemize}
	Non così banale è invece l'assegnazione delle tre matrici di covarianza. Valori bassi potrebbero concentrare le campane in regioni molto piccole
	dello spazio di stato delle risorse, lasciando vaste zone a concentrazione infinitesima e verosimilmente vanificando la complessità di questo approccio.
	Inoltre occorre ponderare la correlazione tra le variabili, per riempire adeguatamente lo spazio di stato $\Psi$ (un reticolo di un parallelepipedo) nel
	caso di problemi con grande disuguaglianza di risorse disponibili.
	Si è seguito un approccio tale per cui, grazie al grado di libertà offerto dalla scelta di $a$, è permesso impostare funzioni gaussiane corrette che
	rispettino la seguente condizione: \\
	sia $b$ il punto di minimo desiderato della curva gaussiana in $\Psi$, allora 
	$$ \log a = \frac{-2\log b}{T^2\,C^{-1}_{11} + TQ\, (C^{-1}_{21} + C^{-1}_{12}) + Q^2\,C^{-1}_{22}} $$
	Ora è permesso settare $C$ a un valore qualsiasi, si è scelto banalmente la matrice identità. Per costruzione pertanto $f_X(x) \geq b \;\;\forall x \in \Psi$.
	
	\maxfig{gaussians.png}{gaussBells}{Esempio di campane \emph{ADD}, \emph{SWAP}, \emph{REMOVE} nello spazio di stato $T, Q$.}
	
	Si osserva infine che, considerato
	\begin{itemize}
	  \item il determinismo \emph{interno} a ciascuna delle tre mosse
	  \item la possibilità che una mossa fallisca, perchè nessuna soluzione tra le sue candidate è $TQ_{OK}$
	\end{itemize}
	invocare la mossa dello stesso tipo di una appena eseguita che abbia restituito un valore di fail è completamente inutile.
	Un piccolo improvement in questi casi è ottenuto con una struttura denominata informalmente \emph{meta taboo list}, che fondamentalmente tiene
	memoria della possibile ultima tipologia di mossa fallita. Il non determinismo dell'estrazione nella logica \emph{STM} permetterà di selezionare
	velocemente un'altra mossa da invocare.
%~~~~~~~~~~~~~~~~~~~~~~~~~~~~~~~~~~~~~~~~~~~~~~~~~~~~~~~~~~~~~~~~~~~~~~~~~~~~~~~~~~~~~~~~~~~~~~~~~~~~~~~~~~~~~~~~~~~~
\section{Livello \emph{local optimum search}}
Rappresenta la parte meno evoluta e responsabile di decisioni dell'algoritmo. Viene invocato una volta che \emph{STM} ha deciso quale mossa intraprendere: 
a quel punto l'intorno è definibile esattamente ed esplorabile. Si è adottato un paradigma \emph{best improvement}, perciò la ricerca nel medesimo sarà esaustiva.
La valutazione delle possibili soluzioni $(x\prime,g\prime)$ è limitata alle sole che:  
\begin{itemize}
  \item non sono la soluzione corrente $(x\prime,g\prime) \neq (x,g)$
  \item non sono vietate dalla \emph{taboo list}.
  \item sono vietate dalla \emph{TL} ma soddisfano i \emph{criteri di aspirazione}.
  \item sono feasible e quindi $TQ_{OK}$.
\end{itemize}
Quindi verrà mantenuta una cache del massimo $(\hat{x}\prime,\hat{g}\prime)$ trovato, confrontandola sempre con le nuove soluzioni generate.
