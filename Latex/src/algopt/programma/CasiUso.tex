\section{Casi d'uso}
%% CASO USO A - ANALISI SINGOLA ISTANZA
\begin{usecase}
	\addtitle{Caso d'uso A}{Analisi di singola istanza} 
	\addfield{Scope:}{System-wide} %the system under design
	%Level: "user-goal" or "subfunction"
	\addfield{Goal:}{ottenere una risoluzione metaeuristica di un certo problema UCARPP.}
	\addfield{Attore primario:}{l'utente finale}
	%Stakeholders and Interests: Who cares about this use case and what do they want?
	%\additemizedfield{Stakeholders and Interests:}{
	%	\item Stakeholder 1 name: his interests
	%}
	\addfield{Precondizioni:}{
		\begin{itemize}
		  \item un file locale nel quale sia correttamente espressa la configurazione
		  di un problema.
		\end{itemize}}
	\addfield{Postcondizioni:}{un report corretto \footnote{una struttura dati contenente:
		\begin{itemize}
		  \item profitto totale
		  \item costo totale, come somma dei tempi utilizzati dai veicoli
		  \item lista di $K$ \emph{percorsi}, ognuno descritto come:
		  \begin{itemize}
		     \item lista dei lati percorsi, nella forma \; $(v_1,\,v_i)\; (v_i,\,v_j)\; (v_i,\,v_j)\; \ldots \; (v_m,\,v_1)$
		     \item lista dei lati serviti, quindi un sottoinsieme della lista precedente, nel medesimo formato
		     \item profitto del percorso
		     \item costo temporale del percorso
		     \item domanda totale servita nel percorso
		     \end{itemize} 
		\end{itemize}},
		il cui formato è adattato a quello utilizzato nelle benchmark.}
	
	\addscenario{Scenario principale di successo:}{
		\item l'utente chiama il programma indicando quale problema vuole risolvere con quanti veicoli.
		\item il programma effettua la sua analisi e restituisce un report testuale di quanto trovato.
	}
	\addscenario{Estensioni:}{
		\item[1.a] la sinopsi è invalida:
			\begin{enumerate}
			\item[1.] il sistema informa in modo generale dell'errore
			\item[2.] il programma termina senza successo.
			\end{enumerate}
		\item[1.b] il file indicato non è presente o è in un formato non valido:
			\begin{enumerate}
			\item[1.] il sistema informa del file non trovato
			\item[2.] il programma termina senza successo.
			\end{enumerate}
		\item[2.b] è implementato il modulo Matlab/Octave:
			\begin{enumerate}
			\item[1.] viene scritto uno script per graficare la serie storica della
			funzione obiettivo e della selezione delle mosse.
			\end{enumerate}
	}
\end{usecase}

%% CASO USO B - ANALISI STATISTICA
% \begin{usecase}
% \addtitle{Caso d'uso B}{Analisi statistica} 
% \addfield{Scope:}{system-wide}
% \addfield{Goal:}{ottenere statistiche sulle prestazioni dell'algoritmo su tanti problemi UCARPP.}
% \addfield{Attore primario:}{i designer della metaeuristica.}
% \addfield{Precondizioni:}{
% 	\begin{itemize}
% 	  \item una directory del filesystem realmente esistente.
% 	  \item un intero $K$ tra i valori {2,3,4}.
% 	\end{itemize}}
% \addfield{Postcondizioni:}{un report statistico corretto 
% 	\footnote{una struttura dati contenente:
% 		\begin{itemize}
% 		  \item i due vettori degli scostamenti relativi dalle due benchmark
% 		  \item i due massimi dei suddetti vettori
% 		  \item i due minimi dei suddetti vettori
% 		  \item i due valori medi dei suddetti vettori
% 		  \item le due varianze dei suddetti vettori
% 		\end{itemize}},
% 		il cui formato è adattato a quello utilizzato nelle benchmark.}
% \addscenario{Scenario principale di successo:}{
% 	\item l'utente chiama il programma indicando la directory dove ricercare le istanze di problema e 
% 		  con quanti veicoli risolverle.
% 	\item su ciascun file presente nella directory viene eseguito il caso d'uso A. 
% 	\item il programma termina le analisi e restituisce un rapporto testuale di quanto eseguito.
% }
% \addscenario{Estensioni:}{
% 	\item[1.a] la sinopsi è invalida:
% 		\begin{enumerate}
% 		\item[1.] il sistema informa in modo generale dell'errore
% 		\item[2.] il programma termina senza successo.
% 		\end{enumerate}
% 	\item[2.a] almeno un file indicato non è presente:
% 		\begin{enumerate}
% 		\item[1.] il sistema informa del file non trovato
% 		\item[2.] il programma termina senza successo.
% 		\end{enumerate}
% 	\item[3.a] è implementato il modulo Matlab:
% 		\begin{enumerate}
% 		\item[1.] viene graficato un istogramma degli scostamenti relativi dalle benchmark.
% 		\end{enumerate}
% 	}
% \end{usecase}