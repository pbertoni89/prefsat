\section{Individuazione dei blocchi procedurali d'interesse}
La complessità temporale di \textbf{Pref} dipende fortemente, oltre che dalle
dimensioni del problema, dal numero di chiamate ai seguenti algoritmi:
\begin{itemize}
  \item \emph{Grounded}
  \item \emph{SCCSEQ}
  \item \emph{Boundcond}
  \item \emph{PrefSAT}
  \item \emph{Pref} stesso.
\end{itemize}
L'analisi quindi è stata strutturata per fornire due indici:
\begin{itemize}
  \item $\eta$ - una misura del tempo ``\emph{tic-toc}'' per ciascuna chiamata a
  una delle suddette procedure, più il tempo totale di esecuzione del programma;
  \item $\xi$ - il numero di chiamate a ciascuna procedura, nell'arco
  dell'intera esecuzione.
\end{itemize}
Mentre il primo indice è atto a relazione la dimensione del problema al tempo in
sè, ed è fortemente mediato dalle prestazioni della macchina di esecuzione, il
secondo è una misura molto buona delle differenze tra la versione \emph{base} di
\textbf{Pref} e la versione \emph{improved}, implementazione dello studio
presentato nel capitolo \ref{chap:opt}. Chiamando il programma sullo stesso
problema nelle due modalità, infatti, è rapidamente visibile il risparmio in
termini di chiamate introdotto dalle ottimizzazioni. Si noti che \emph{SCCSEQ}
non è stato introdotto nell'indice $\xi$ siccome non è sottoposto ad
ottimizzazioni, pertanto il numero delle sue chiamate è sempre pari a quello
relativo a \emph{Pref}.

 Entrambi gli indici sono sempre restituiti dal programma
sullo standard output.

\section{Verifica di correttezza: gli oracoli} \label{sec:oracles}
Per testare innanzitutto i requisiti funzionali ci si è avvalsi dell'oracolo
\emph{ASPARTIX} disponibile come web service dall'\emph{URL}
\url{http://rull.dbai.tuwien.ac.at:8080/ASPARTIX}. Si noti che la rete di
Facoltà dovrebbe bloccare la porta 8080. L'applicazione oracolo, oltre a
confermare gli output dei tre esempi forniti insieme alle specifiche, ha 
validato anche altri test, generati secondo il metodo descritto in \ref{sec:graphgen}.

\section{Metodo di misura del tempo}
La scelta è ricaduta sulla libreria \textbf{chrono} del $C_{++}$, che
costituisce anche un \emph{subnamespace} \emph{std::chrono}.
I concetti principali implementati sono tre:
\begin{itemize}
  \item le \emph{durate temporali} misurano lassi di tempo, fino all'ordine dei
  millisecondi. In questa libreria sono rappresentate come oggetti della classe
  template \emph{duration}, che accoppiano una rappresentazione di conto e una
  precisione di periodo (\emph{e.g.}, 10 millisecondi ha 10 come
  rappresentazione di conto e 'millisecondi' come precisione di periodo.)
  \item gli \emph{istanti temporali} sono riferimenti a dei punti precisi
  sull'asse temporale, espressi come una \emph{duration} da un punto fissato per
  convenzione.
  \item un \emph{clock} è un framework collegante un \emph{istante temporale} a
  un'istante del mondo reale.
\end{itemize}
La libreria offre almeno tre clocks diversi per esprimere il tempo corrente come
un valore numerico: \emph{system\_clock}, \emph{steady\_clock} e
\emph{high\_resolution\_clock}, a miglior precisione e che sarà per questo
utilizzato nelle benchmark.

Si noti che per utilizzare \textbf{chrono} è obbligatoria una compilazione
conforme allo standard $C_{++}11$.
%% ~~~~~~~~~~~~~~~~~~~~~~~~~~~~~~~~~~~~~~~~~~~~~~~~~~~~~~~~~~~~~~~~~~~~~~
\section{Configurazione del calcolatore}
Le benchmarks sono state eseguite su un notebook così caratterizzato:
\begin{itemize}
  \item \textbf{OS} Linux Ubuntu 10.04 con istruzioni a 32 bit
  \item \textbf{CPU} Intel Pentium(R) Dual-Core CPU T4200 , clock 2.00GHz e
  tecnologia \emph{Hyper Threading}
  \item \textbf{RAM} 4GB DDR3, di cui 3 disponibili all'utente.
  \item \textbf{Compiler} Gnu C Compiler (gcc) in versione $C^{++}11$ e
  ottimizzato.
\end{itemize}
%% ~~~~~~~~~~~~~~~~~~~~~~~~~~~~~~~~~~~~~~~~~~~~~~~~~~~~~~~~~~~~~~~~~~~~~~
\section{Un generatore di problemi} \label{sec:graphgen}
Le tre istanze di prova allegate alle specifiche del problema vengono risolte in
un tempo dell'ordine dei 10 millisecondi, con una varianza elevata, e quindi non
sono utilizzabili per una profilazione consistente della complessità temporale.
Per generare grafi più onerosi è stato implementato un piccolo applicativo
indipendente, che libera l'analista dallo sforzo di pensare e codificare
argumentation framework anche di migliaia di nodi o archi.
I parametri richiesti sono:
\begin{itemize}
  \item numero $N$ di \emph{Arguments}, ovvero nodi del grafo. Verranno
  automaticamente denominati $n_i,\;\; i\in[1,N]$.
  \item il numero massimo $\overline{A}$ di \emph{Arguments} che attaccano o
  sono attaccati da ogni \emph{Argument}. Ricordando che un grafo orientato di $N$
  nodi completamente magliato è dotato di $N(N-1)$ archi, il valore
  $\frac{N(N-1)}{N} = N-1$ costituisce un upperbound per $\overline{A}$. Un
  grafo denso ``a metà'' è chiaramente caratterizzato da un $\overline{A} =
  \frac{N}{2}$.
  \item nome del file in formato \emph{dl} dove il \emph{framework} verrà
  salvato.
  \item $\{graph, dot, nograph\}$ per indicare se si desidera anche un
  output grafico in formato jpg, rappresentante il grafo. È qui utilizzato, come
  nella documentazione del sorgente scritta con \textbf{Doxygen}, il linguaggio
  \textbf{dot}. Nel caso ``dot'' verrà scritto solo il file dot senza generar
  l'immagine.
\end{itemize}

Applicativo \textbf{graphGenerator} e relativo sorgente vengono allegati al
progetto finale. Vengono inoltre forniti degli esempi casuali generati con
esso, utilizzati nei test che saranno presentati a breve. I loro nomi rispettano
la sintassi $N$\emph{\_}$\overline{A}$\emph{trial.dl}, dove trial è una lettera
$a,b,\ldots n$ che distingua i problemi di uguale dimensione generati.
È stato deciso inizialmente di esplorare gli ordini di grandezza di $N$ da $10$ a $10^4$, ma, come si vedrà,
già il passaggio da $10^2$ a $10^3$ si mostrerà proibitivo.

 Se si suppone una complessità $O(1)$
per elaborare ciascun nodo, esclusa dalla chiamata la gravosissima elaborazione dell'immagine, la complessità di graphGenerator è
chiaramente $O(N\overline{A})$. Lo spazio richiesto per la memorizzazione dei
grafi si comporta anch'esso linearmente rispetto alle singole dimensioni $N,
\overline{A}$, come suggerito in figura \ref{fig:filesize}.

\singlefig{dl_filesize.png}{Dimensione su disco delle istanze
del problema, al variare di $N$ (ascisse) e con due $\overline{A}$ fissate
relativamente a $N$.}{12.0}{filesize}
%% ~~~~~~~~~~~~~~~~~~~~~~~~~~~~~~~~~~~~~~~~~~~~~~~~~~~~~~~~~~~~~~~~~~~~~~
\section{Benchmarks sui test generati}
\textbf{Note:} un trial contrassegnato dal ``$^+$'' rappresenta l'esecuzione migliorata su quel grafo.
 Nella parte destra della tabella, per non appesantire la lettura sono stati omessi
i valori invariati tra ciascuna coppia di esecuzioni. I tempi sono espressi in millisecondi.
\begin{table}
    \begin{tabular}{lll|lllll}
    $N$&$\overline{A}$ & \emph{trial} & \emph{total time} & \emph{\#Grounded} & \emph{\#Boundcond} & \emph{\#PrefSAT} & \emph{\#Pref} \\ \hline \hline
    
    \emph{esempio} & ~ & $1$ 	& 4          & 1          & 5           & 5         & 1      \\
    ~   & ~            & $1^+$	& ~          & ~          & 2           & 4         & ~      \\ \cline{3-8}
    ~   & ~            & $2$ 	& 1          & 1          & 1           & 1         & 1      \\
    ~   & ~            & $2^+$	& ~          & ~          & 0           & 0         & ~      \\ \cline{3-8}
    ~   & ~            & $3$ 	& 3          & 1          & 3           & 3         & 1      \\
    ~   & ~            & $3^+$	& ~          & ~          & ~           & 2         & ~      \\ \hline \hline
    
    50 & 25		       & $a$   		& 100      & 1          & 1           & 1         & 1 \\
    ~   & ~            & $a^+$		& 95         & ~          & ~           & ~ 		& ~ \\ \cline{3-8}
    ~   & ~            & $b$    	& ~          & ~          & ~           & ~         & ~ \\
    ~   & ~            & $b^+$		& ~          & ~          & ~           & ~ 		& ~ \\ \cline{3-8}
    ~   & ~            & $c$   		& ~          & ~          & ~           & ~         & ~ \\
    ~   & ~            & $c^+$		& ~          & ~          & ~           & ~ 		& ~ \\ \hline \hline
    
    50  & 49          & $a$   		& 63000      & 1          & 1           & 1         & 1 \\
    ~   & ~            & $a^+$		& ~          & ~          & ~           & ~ 		& ~ \\ \cline{3-8}
    ~   & ~            & $b$    	& ~          & ~          & ~           & ~         & ~ \\
    ~   & ~            & $b^+$		& ~          & ~          & ~           & ~ 		& ~ \\ \cline{3-8}
    ~   & ~            & $c$   		& ~          & ~          & ~           & ~         & ~ \\
    ~   & ~            & $c^+$		& ~          & ~          & ~           & ~ 		& ~ \\ \hline \hline
    
	100 & 50           & $a$   		& 900     	 & 1          & 1           & 1         & 1 \\
    ~   & ~            & $a^+$		& 850        & ~          & 0         	& ~ 		& ~ \\ \cline{3-8}
    ~   & ~            & $b$    	& 1150       & 1          & 1           & 1         & 1 \\
    ~   & ~            & $b^+$		& 1160       & ~          & 0           & ~ 		& ~ \\ \cline{3-8}
    ~   & ~            & $c$   		& 550        & 1          & 1           & 1         & 1 \\
    ~   & ~            & $c^+$		& 560        & ~          & 0           & ~ 		& ~ \\ \hline \hline
    
    100 & 99           & $a$   		& 6000       & 1          & 1           & 1         & 1 \\
    ~   & ~            & $a^+$		& 6000       & ~          & 0           & ~ 		& ~ \\ \cline{3-8}
    ~   & ~            & $b$    	& 4150       & 1          & 1           & 1         & 1 \\
    ~   & ~            & $b^+$		& 4150       & ~          & 0           & ~ 		& ~ \\ \cline{3-8}
    ~   & ~            & $c$   		& 4700       & 1          & 1           & 1         & 1 \\
    ~   & ~            & $c^+$		& 4700       & ~          & 0           & ~ 		& ~ \\ \hline \hline
    
    200 & 100          & $a$   		& 63000      & 1          & 1           & 1         & 1 \\
    ~   & ~            & $a^+$		& 60000      & ~          & 0           & ~ 		& ~ \\ \cline{3-8}
    ~   & ~            & $b$    	& 34000      & 1          & 1           & 1         & 1 \\
    ~   & ~            & $b^+$		& 33000      & ~  		  & 0           & ~			& ~ \\ \cline{3-8}
    ~   & ~            & $c$    	& 17900      & 1          & 1           & 1         & 1 \\
    ~   & ~            & $c^+$		& 17800      & ~          & 0           & ~ 		& ~ \\ \hline
    
    \end{tabular}
\end{table}


\section{Considerazioni finali}
In tabella sono stati riportati solo i tempi totali e non alle singole chiamate,
ma si registra che il tempo utilizzato da \emph{PrefSAT} domina fino al 90\%
della durata totale di ogni esecuzione. È altresì la più onerosa in termini di
memoria principale allocata, e siccome è codice scritto da terzi si sospetta che
presenti varie memory leak. Si può dunque affermare che l'ottimizzazione più rilevante di ogni altra sia quella che
eviti la chiamata al problema SAT, peraltro notorialmente \emph{NP-hard}.

Moltissime volte, mentre \emph{Pref} gira in versione migliorata, viene risparmiata la chiamata a \emph{Boundcond}
poichè il relativo \emph{SCC} non ha nessun padre. Questo è possibile che sia legato alla generazione puramente pseudocasuale
dei grafi, che non cerca di costruire nessun percorso arbitrariamente lungo sui nodi, ma si limita a computare un certo numero
di probabilità di collegamento. Versioni migliorate della struttura di test potrebbero incorporare un'euristica della ``connessione''
del grafo generato; il trascurabile vantaggio del miglioramento su \emph{Boundcond} non ne giustifica per ora lo sforzo.

Si noti inoltre la grande differenza nel numero di chiamate, tra tre istanze note come \emph{esempio1}, \emph{esempio2} e \emph{esempio3},
e invece i grafi generati casualmente. È evidente che argumentation framework strutturati senza una logica apparente rimangono alquanto
sconnessi e, oltre a non avere interesse reale nella loro risoluzione, ne hanno ben poco dal punto di vista dell'indice $\xi$. Qualora fossero
disponibili istanze di grafi di dimensioni considerevoli e ben strutturati, saremo lieti di incorporarli in una seconda e più informativa versione 
del testing. 

Parimenti si può dire che le istanze generate casualmente, dato il numero sempre minimo di chiamate alle varie procedure, costituiscono una sorta
di \emph{lower bound} sulle prestazioni di \emph{Pref} in generale.